My time as a PhD student and the time leading up to this research was shaped by many and it is important 
to me to take the time to express my gratitude for those who personally and academically made this work possible.

\begin{center} \textbf{Advisor} \end{center}

I would like to first acknowledge my advisor Chris Tully. He has guided my PhD experience
by directing me towards projects where I could succeed. I thank him  for always making himself available and
for making me feel supported and welcome within the Princeton group.  I owe much of my success and happiness 
as a PhD student to the intellectual freedom he gave me to explore my ideas. I have always 
felt, and continue to feel, that Chris values my happiness and success as a person above all
else. Chris is an exemplary physicist and role model who leaves me continually humbled
 by our interactions. Princeton is lucky to have him as a teacher and researcher. 

\begin{center} \textbf{Mentors} \end{center}

I was fortunate to have  the pleasure of working with Maurizio Pierini for more than
 7 years, a period from when 
I was an undergraduate at Caltech till my last days at CERN. Maurizio worked with me through
the details of the analysis from its conception
to the final approval. Week by week, and espresso by espresso, we pushed the analysis through
the final approval process.  I'd like to thank Maurizio for being my advocate. When data samples were delayed in 
production, he would reach out to management on our behalf. When I applied for fellowships, he 
was always willing to write a recommendation. He is an esteemed colleague, a mentor and 
a friend. I wish him the best.

Toyoko Orimoto began mentoring me when I was an undergraduate at Caltech
performing a search for diphoton search with Yousi Ma. During my first visit to CERN, she was kind
 enough to invite myself and fellow hungry students to her apartment in Servette for vegan dinners. 
She made me feel a part of the CERN and high energy physics community. I express my gratitude 
for the time she spend editing 
my NSF Fellowship proposal and graduate school applications. Toyoko always actively kept me in her thoughts. 
She would nominate me for talks and forward job opportunities.
I am proud to have seen her attain a professorship position at Northeastern. Her students will inevitably 
know the same nurturing, caring, and intelligent physicist I have had the joy of working with. I value her friendship and am excited to see
 her continued success as a professor.

\begin{center} \textbf{Princeton} \end{center}

When I first arrived to CERN and joined the Princeton Group, I will remember Halil Saka for picking me up from the Geneva airport. 
I'd like to thank him for some of the most intense and enjoyable discussions I had
at CERN. He is at his best acting as my
my de facto table partner at conferences. When I think of Halil I think 
futbol, Clubhouse nachos, and trying to calculate longitudinal impact parameters from track variables by hand. Halil is a one of a kind physicist. 

I also had the pleasure of working with Andrzej Zuranski.
The theoretical community had significant interest in what
he had done, especially the inclusive nature of the search. I was drawn to continue his search after he graduated and used
the triggers and variables of his analysis as a starting point for improvement at 13 TeV.
The documentation and discussions with Andrzej were integral to the decisions made in the
early stages of the analysis. I thank him for enabling me to perform this study. 

Thank you Phil Hebda for pushing me to learn French. His performance as the 1 km leg in the
 CERN relay will not be forgotten. Go $\Delta F$! 
His dedication to his work and his hobbies are admirable. 

Additionally, I would like to thank Xiaohang Quan for aiding my initial work on the diphoton triggers,
Paul Lujan for useful discussion on long lived monte carlo generation, and Dan Marlow for serving as my second
reader for this work. Many thanks to Ted Laird and Edmund Berry for their advice and wisdom on navigating CERN, CMS, 
and Princeton.

In addition to my colleagues in within the CMS group I would like to thank Professors Mariangela Lisanti and Joshua Shaveitz for serving
on my thesis committee.

My time at Princeton would not have been the same without my close physics department friends in the 2012 cohort. 
Our preliminary exams were a formative time in my life and I know it would not
 have been the same without you. 
Aitor Lewkowycz you are an impressive theorist and I look forward to following
your career. I fondly remember the time you came to visit me at CERN. 
 Zachary Sethna, your love of problem solving, reading, and politics continues to keep me sharp. The 
Z-bar will forever live on in the minds of those who enjoyed your generosity and welcoming demeanor. Joaquin Turiaci, you
are incredibly patient and insightful. Many of the finer points in this thesis are the product
of our discussions. Great men are born on September 27th. Aaron Levy, I respect your detailed approach to problem solving. I will remember you best for our work on QFT
problem sets together and arguing with Zach about politics. 
You were a like minded personality during an important time of my life,
 thank you for being my friend. Dave Zajac, you are a machine. I will not forget 
our early morning gym work outs spent discussing prelim problems.
I admire your dedication to your lab and respect your ability as an experimental physicist. Keep on climbing. I could not possibly leave out KL Moscato 
who provided a much needed outlet for me from physics to talk about dance and yoga. 
Thank you for your hospitality and your listening ear. You are tough as nails 
and I'm excited for you moving on from ARB. Mattia you are hilarious and keep 
me laughing, please never change.

\begin{center} \textbf{CERN} \end{center}

I would like to thank Daniele Del Re, Slava Valouev, and Oliver Muchmuller for 
serving as the Exotica group conveners on this analysis.
Additionally, I thank Ted Kolberg and Wells Wulsin for serving as the
long-lived subgroup conveners. I appreciate the time spent by Luca Scodellaro,
 Kevin Stenson, Matt Walker, and Bora Isildak as the ARC members of the analysis. 
Many thanks to fellow CERN affiliates who helped me along the way including: Zeynep Demiragli, Mia Tosi, Matt Strassler, Steve Mrenna, Juliette Alimena, Jamie Antonelli,
Bingxuan Liu, Luca Pernie, Emanuele Di Marco, Josh Bendavid, 
Nadir Daci, and many more.

I also had the great pleasure of living with \textit{two of the best} 
roommates I could have asked for. Brad Axen and Russell Smith, you were the
Americans I needed in my life. Thank you for the adventures, the tribulations,
the laughter, and the numerous catastrophes that we endured at Avenue de S\'echeron 7, 5\`eme \'etage. I am happy to see
 both of you move on from Geneva and restart
your lives in the US. I want to smile and cry every time I think about that 
apartment. You are both bright, good natured physicists. Thank you for being 
my friends.

\begin{center} \textbf{Caltech} \end{center}

I began my work on the CMS experiment with the Caltech group working with Harvey Newman and Maria Spiropulu. I am forever grateful  for the impact this had on my 
trajectory in science. Harvey and Maria have always been nurturing
and encouraging in my pursuit of doing research. 
I admire Harvey's love of science and his commitment to advocating
in Washington DC for fundamental science. My first summer, a Caltech senior, Andy Yen, taught me the basics of experimental physics 
and how to work with ROOT. I want to thank him for remaining a friend and advising
 me on career decisions while he manages his successful startup: Proton Mail. I must
also express my gratitude to Yousi Ma for answering day to day questions and advising
me through the graduate school process. 
Yousi is a patient and kind person as well as a scientist. I was lucky to have him
as a mentor. Thank you Chris Rogan for counseling me on academic life and for some
of the most educational discussions on particle physics I had in my time as a student. I consider you both a role model friend and physicist. 
I found inspiration in your research and am proud to have known you personally. I would additionally like to
thank members of the Caltech group who supported me in my research: Marat Gataullin, 
Vladamir Litvin, Vladlen Timciuc, and Yong Yang.

\begin{center} \textbf{Family and Friends} \end{center}

Mom and Dad, you have always stood behind my pursuit to go to college and graduate school. Thank you for constant words of encouragement and financial support. 
You've raised me to be who I am today and I can never repay the sacrifices
you've made to get me here.

I also want to thank those friends who have supported me from afar: Thompson and Kristina Heavey, Adam Khan, Arda Antikacioglu, Paul Fleiner, Barbara Gao,
 Kevin Shaprio, Helen Jing, and Colin Ho. I always knew when I came back to the US you had
a place for me to stay and reminisce. Good friends are hard to find, but easy to keep.

A special thank you goes out to  Arda. Through the toughest times and the best of times you have always been on the other 
side of the phone to break it down. Our friendship is a high priority in my life and I wish you the best. You are always near to my thoughts. This ones for you. 

\begin{center} \textbf{Summary} \end{center}

While my PhD has had it's share of difficult and isolating times, I have had the pleasure and
the privilege of being surrounded by a revolving door of smart and engaging personalities at both Princeton and CERN. Physics would not have been the same to me without
you in it.

I would like to end with a quote from an interview with Richard P. Feynman. 
I turn to this quote for inspiration in difficult times to be reminded myself
 why I do physics. Thank you for your words. They are a reminder to keep
my mind open to the world not as I see it, but as it chooses to reveal itself. 

\begin{quote} 
\textit{``People say to me: are you looking for the ultimate laws of physics? No, I'm not. I'm just looking to find
out more about the world and if it turns out there is a simple ultimate law that explains everything--so be it. That would be very nice to discover. 
If it turns out its like an onion with millions of layers and we're just sick and tired of looking at the layers.
Then that's the way it is. But whatever way it comes out its nature that's there and she's going to come out the way she is. 
Therefore when we go to investigate it, we shouldn't pre-decide what we want to find out besides to find
out more about it.  You see -- one thing is, I can live with doubt, and uncertainty 
and not knowing. I think its much more interesting to live not knowing than to have answers that might be wrong. 
I have approximate answers and possible beliefs, and different degrees of certainty about different things, but I'm
not absolutely sure of anything and there are many things I don't know anything about. 
But I don't have to know an answer. I don't have to. I don't feel frightened by not knowing things.''}
\end{quote}
\begin{center} ---Richard P. Feynman \end{center}
\begin{center}
\includegraphics[width=.55\textwidth]{pics/feynman_pic}
\end{center}


\newpage

This material is based upon work supported by the National Science Foundation Graduate Research Fellowship
 under Grant No. DGE-1656466. Any opinion, findings, and conclusions or recommendations expressed in this material are those of the authors(s) and do not necessarily reflect the views of the National Science Foundation.

%% J'amerias dire merci a mon bonne ami Mia Mehic. Je sais la vie entre toi and moi etais dificile a cause de la barrier de langue, 
%% mais aujourdhui nous sommes plus bonnes avec les deux. Tu me manques et j'espere tu vas profiter chaque jour comme avant 
%% j'ai parti. T'es quelqun special a moi et j'espere on peut se voir dans l'avenir. La langue francaise et la suisse vas rester dans
%% une place tellement speciale dans mon coeur.

