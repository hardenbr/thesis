\chapter{Theory}

\section{Standard Model of Particle Physics}

\subsection{Quantum Field Theory}

In classical lagrangian kinematics, the time evolution of some coordinate $q$ is determined via the principle of minimal action. 
\begin{equation}
S[q(t)] = \int_{t_1}^{t_2} L(q,\frac{dq}{dt},t) dt
\end{equation}
where $S$ is a functional of the time dependent coordinate $q(t)$. 

\subsection{Symmetries in the Standard Model}

The fundamental forces and their interactions arise from the symmetries preserved by the standard model lagrangian. 

\subsection{Fundamental Fields of the Standard Model Lagrangian}



\subsection{Sectors of the Standard Model Lagrangian}

The Standard Model of particle physics consists of a quantum field theory lagrangian with four sectors:

\begin{equation}
\mathcal{L}_{SM} = \mathcal{L}_{Gauge} + \mathcal{L}_{Fermion} + \mathcal{L}_{Higgs} + \mathcal{L}_{Yukawa}
\end{equation}

and three gauge group symmetries: $U(1)_Y$ hypercharge, 
$SU(2)_L$ left chiral and $SU(3)_c$ color. All standard model particles transform as a multiplet of
 $SU(3) \times SU(2)_L \times U(1)_Y$.

The gauge sector consists of the field stress energy tensor of the 3 corresponding types of gauge bosons:
 $G^i$ (gluons of the color force), $W^i$ ($W$'s of the weak force) and $B$ ($B$'s of the weak hypercharge). 
Here the index $i$ enumerates their multiplicity. There are 8 gluons, 3 $W$'s and a single B. 

\begin{equation}
\mathcal{L}_{Gauge} = - \frac{1}{4} F_{\mu\nu}^{i} F^{\mu\nu i} =  - \frac{1}{4} G_{\mu\nu}^{i} G^{\mu\nu i} - \frac{1}{4} W^{i}_{\mu\nu} W^{\mu\nu i} - \frac{1}{4} B_{\mu\nu}B^{\mu\nu} 
\end{equation}

where the double scripts correspond to the antisymmetric commutator, that is:

\begin{equation}
X_{\mu\nu}^i = [ D_{[\mu} X_{\nu]}^i  = D_\mu X_\nu^i - D_\nu X_\mu^i - g f_{ijk} X_\mu^j X_\nu^k
\end{equation}
where the $D_\mu$ terms correspond to the covariant derivative and the $f_{ijk}$ are the corresponding structure constants for the algebra. and $g$
the coupling constant. In full, the field stress tensor terms are:

\begin{align*}
G_{\mu\nu}^i &=  D_\mu G_\nu^i - D_\nu G_\mu^i - g_s f_{ijk} G_\mu^j G_\nu^k\\ 
W_{\mu\nu}^i &=   D_\mu W_\nu^i - D_\nu W_\mu^i - g \epsilon_{ijk} W_\mu^j W_\nu^k\\ 
B_{\mu\nu}^i &=  D_\mu B_\nu^i - D_\nu B_\mu^i
\end{align*}

The fermion sector consists of the kinetic energy terms for each quark (up and down types) and leptons (lepton, neutrinos) in the standard model.
The left handed quarks transform as an SU(2) doublet:

\begin{equation}
q^0_{mL\alpha} = \left( \begin{array}{c} u_{m\alpha}^0  \\ d_{m\alpha}^0 \end{array} \right)_L 
\end{equation}

where the subscript $m$ denotes the family (1st, 2nd and 3rd generation) and $\alpha$ denotes the color charge (red, green, and blue).

\begin{equation}
l_{mL} = \left( \begin{array}{c} \nu_{m}^0  \\ e^{-,0}_{m} \end{array} \right)_L 
\end{equation}

As the $SU(2)_L$ symmetry only acts on the left handed fermions we further separate the fermion sector into left and right components:

\begin{equation}
\mathcal{L}_{fermion,L} = \bar{q}^0_{mL} i \gamma^\mu D_\mu q^0_{mL} + \bar{l}^0_{mL} i \gamma^\mu D_\mu l^0_{mL}  \\
\end{equation}

\begin{equation}
\mathcal{L}_{fermion,R} =  \bar{u}^0_{mR} i \gamma^\mu D_\mu u^0_{mR} 
+ \bar{d}^0_{mR} i \gamma^\mu D_\mu d^0_{mR} + \bar{e}^0_{mR} i \gamma^\mu D_\mu e^0_{mR} + \bar{\nu}^0_{mR} i \gamma^\mu D_\mu \nu^0_{mR}
\end{equation}

The higgs sector consists of terms related to the single scalar field $\phi$

\subsection{Electroweak Symmetry Breaking}

\subsection{The Narrow Width Approximation}

\section{Supersymmetry}

\subsection{Electroweak Symmetry Breaking in Supersymmetric Theories}

% neutralinos, charginos, gauginos, 

\section{Origins of Long-lived Signatures}

\subsection{Standard Model Particles with Long Lifetimes}

The standard model already includes a variety of particles that can generate displaced vertices (Table \ref{tab:mesons} Table \ref{tab:baryons}). 
For example, $B^0 \rightarrow J/\psi K^{*0}$ with $K^{*0} \rightarrow K^+\pi^-$ generates a 4 track vertex. Such a vertex is commonly utilized 
in b-tagging. Of particular interest to single displaced jet identification outside of the b-tagging regime are charge neutral SM particles
decaying to charged particles with a few centimeter lifetime: $\Lambda^0$, $K_S^0$. Such particles would have no track
leading to the primary vertex and vertices far outside the b lifetime. The most relevant of processes being:

\begin{enumerate}
\item $K_s^0 \rightarrow \pi^+\pi^-$ 69\% of all $K_s^0$ decays 
\item $\Lambda^0 \rightarrow p \pi^-$ 64\% of all $\Lambda^0$ decays 
\end{enumerate}

Jets containing prompt and non-prompt $K_s$ and $\Lambda^0$ will contain tracks with large impact parameters, 
and large impact parameter significance. When a vertex is fit to the matched tracks we expect small track multiplicity relative 
to the GeV to TeV   long-lived particles this identification targets. It is important to separate this contribution from
the detector effects like nuclear interactions.

\begin{table}
\begin{center}
\begin{tabular}{cccccc}
\hline
\textbf{Name}  & \textbf{Content}                                    & \textbf{Particle}    & \textbf{mass} (MeV) & $\tau_{0}$ (sec)  & c$\tau$ (cm)         \\
\hline 
Pion & $u\bar{d}$                                  & $\pi^{\pm}$ & 139        & $2.6 \times 10^{-8}$       & $7.8 \times 10^{2}$  \\
\hline 
Kaon  & $u\bar{s}$                                 & $K^{\pm}$   & 497        & $     1.23 \times 10^{-8}$ & $3.7 \times 10^2$    \\
K Short  & $\frac{1}{\sqrt{2}}(d\bar{s} - s \bar{d})$ & $K^0_{s}$   & 497        & $0.896 \times 10^{-10}$    & 2.68                 \\
K Long  & $\frac{1}{\sqrt{2}}(d\bar{s} + s \bar{d})$ & $K^0_L$     & 497        & $5.1\times 10^{-8}$        & $1.5 \times 10^3$    \\
\hline
D & $c\bar{d}$                                 & $D^{\pm}$   & 1869       & $1 \times 10^{-12}$        & $3.0 \times 10^{-2}$ \\
\hline
B meson  & $u \bar{b}$                                & $B^{\pm}$   & 5279       & $1.6 \times 10^{-12}$      & $4.8 \times 10^{-2}$ \\
strange B  & $s\bar{b}$                                 & $B^{0}_s$   & 5366       & $1.5 \times 10^{-12}$      & $4.5 \times 10^{-2}$ \\
charmed B  & $c\bar{b}$                                 & $B^{0}_c$   & 6275       & $4.5\times 10^{-13}$       & $1.4 \times 10^{-2}$ \\
\end{tabular}
\end{center}
\caption{Mesons with Lifetimes greater than $10^{-2}$ cm} 
\label{tab:mesons}
\end{table}


\begin{table}
\begin{center}
\begin{tabular}{cccccc}
\textbf{Name}          & \textbf{Content} & \textbf{Particle}      & \textbf{mass} [MeV] &  $\tau_{0}$ [s] & $c\tau_{0}$ [cm] \\
\hline 
Lambda        & $uds$   & $\Lambda^0$   & 1115       & $2.6 \times 10^{-10}$    & 7.8                  \\
bottom Lambda & $udb$   & $\Lambda^0_b$ & 5620       & $1.4 \times 10^{-12}$    & $4.2 \times 10^{-2}$ \\
\hline
Sigma plus    & $uus$   & $\Sigma^{+}$  & 1189       & $8 \times 10^{-11}$      & 2.4                  \\
Sigma minus   & $dds$   & $\Sigma^{-}$  & 1197       & $1.4\times 10^{-10}$     & 4.2                  \\
\hline 
Xi zero       & $uss$   & $\Xi^0$       & 1314       & $4 \times 10^{-13}$      & $1.2 \times 10^{-2}$ \\
Xi minus      & $dss$   & $\Xi^-$       & 1321       & $1.6 \times 10^{-10}$    & 4.8                  \\
charmed Xi +  & $usc$   & $\Xi^{+}_c$   & 2467       & $4.42 \times 10^{-13}$   & $1.3 \times 10^{-2}$ \\
charmed Xi    & $dsc$   & $\Xi^0_c$     & 2471       & $1.12 \times 10^{-13}$   & $3.3 \times 10^{-2}$ \\
bottom Xi     & $dsb$   & $\Xi^-_b$     & 5792       & $1.56 \times 10^{-12}$   & $4.7 \times 10^{-2}$ \\
\hline
bottom Omega  & $ssb$   & $\Omega_b^-$  & 6054       & $1.13 \times 10^{-12}$   & $3.3 \times 10^{-2}$ \\
Omega minus   & $sss$   & $\Omega^-$    & 1672       & $8 \times 10^{-11}$      & 2.4                  \\
\end{tabular}
\end{center}
\caption{Baryons with Lifetimes greater than $10^{-2}$ cm} 
\label{tab:baryons}
\end{table}


Particles of a characteristic lifetime $\tau$ decay with a falling exponential. For reference, 
a table describing the percent of decays that will occur at various distances is shown in Table
 \ref{tab:lifetime}. Even lifetimes 10 and 100 times the size of the tracker, we would still expect
~10\% and 1\% respectively to occur within the tracker. For particles  of lifetime $\lambda$ we
expect 0.6\% to decay beyond $5\lambda$. 


\begin{table}
\begin{center}
\begin{tabular}{ccc}
Distance ($\lambda$) &  Probability of Decay  \\
\hline
0.01               & 1\%                        \\
0.1                & 9.5\%                      \\              
0.25               & 22\%                       \\            
0.5                & 39\%                       \\          
0.75               & 52\%                       \\        
1                  & 63\%                       \\      
1.5                & 77\%                       \\    
2                  & 86\%                       \\  
3                  & 95\%                       \\
5                  & 99.3\%                     \\
\end{tabular}
\end{center}
\caption{ A reference table for the cumulative probability for a particle of lifetime $\lambda$ to have decayed after a given
distance. Distance is in multiples of lambda.}
\label{tab:lifetime}
\end{table}


\subsection{Split-Susy and Naturalness at the LHC}

The expectation of discovering supersymmetry (SUSY) at the TeV scale has been largely motivated
 by arguments based on naturalness. 
Since the mass of the Standard Model Higgs boson is sensitive to the high energy scale where SUSY
 is broken ($m_{SUSY}$), its mass, of order the electroweak scale, $(m_h \approx m_{EW} \ll m_{SUSY})$
 would need to be tuned to order $m_{EW}^2/m_{SUSY}^2$. 
To avoid fine-tuning, we would like  $m_h^2 \approx m_{SUSY}^2 \implies m_{SUSY} \leq$ 1 TeV. 
More specifically, knowing $m_H \approx 125$ GeV we expect light SUSY partners (in particular, light stops)
 near $< 1$ TeV to stabilize the quadratic divergences of 1 loop corrections to the Higgs mass
 [citation:$light_stops$]. 
Unfortunately these scalar partners have yet to be discovered.

It is important to note that the stability of the Higgs boson mass is not the only
 fine-tuning problem in particle physics. 
When the same argument is made for the cosmological constant we arrive at $\Lambda \geq m_{SUSY}^4$, 
where experimentally $\Lambda = 10^{-59}$ TeV$^4$.   
If we use the same SUSY scale as we did for the Higgs mass,
 $m_{SUSY} = 1$ TeV we have a new fine tuning problem of $10^{60}$.

As addressed by Arkani-Hamed and Dimopoulos [citation:$nima_lhc$], many theoretical approaches  have been
 motivated by a natural explanation for the Higgs mass while separately seeking an  explanation
 of the cosmological constant through some other mechanism.
Arkani-Hamed and Dimopoulos propose a reconsideration of naturalness, entertaining the idea that 
fine tuning could have a role to play in beyond the Standard Model physics.
Conceivably, both $\Lambda$ and $m_h$ fine tuning could be resolved by the same mechanism.  
This un-natural model was  further investigated by Giudice and Romanino [citation:$split_susy$]
and dubbed ``split supersymmetry". 

Split SUSY assumes a much higher SUSY scale $m_{SUSY}^2 \gg 1$ TeV where all scalars (excluding the Higgs) 
become very heavy $O(m_{SUSY})$ and the lightest sparticles (Higgsinos and gluinos) are kept at the TeV scale by requiring the lightest neutralino to be a good dark matter candidate. 

Because the scalars are so much heavier, the decay of gluinos through squarks is suppressed.
The characteristic signature of split supersymmetry is thus long-lived gluinos; such processes 
with long lifetimes are rare in the SM.
