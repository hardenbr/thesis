\chapter{Physics Objects\label{ch:objects}}

After data is selected by the trigger, the offline analyses begin with particle identification.
There are no longer any timing limitations, so such identifications can make use of all the detector
information in the event. CMS uses the Particle Flow (PF) algorithm in most analyses, which is
described in Section~\ref{sec:PF}.
As this search centers around the identification of events with both $\Hgg$ and $\Hbb$ decays, the
identification and reconstruction of photons and jets are the first steps taken after the
trigger selects potentially-interesting events,
and this chapter discusses the treatments needed at this stage for both data and MC samples.
Recalling that the sensitivity in the separation between signal and background comes from the
excellend diphoton mass resolution, the identification of two high quality photons is the starting point
and discussed in Section~\ref{sec:photons}. The following step is the identification of two jets
coming from the hadronization of b-quarks and is discussed in Section~\ref{sec:jets}.


\section{Particle Flow\label{sec:PF}}

The PF algorithm recontructs all stable particles in an event from the digitized electronic signals
of all channels in all subsystems~\cite{PFPAS2009,CMS-PAS-PFT-10-001}. These particles include
electrons, photons, charged hadrons, neutron hadrons, and muons, as shown in Figure~\ref{fig:PF}.
From these particles, derived objects are constructed, including jets and missing transverse energy.
The algorithm itself links detector objects created from individual subsystems and groups them into
blocks that are identified with a particle. The detector objects are discussed in
Section~\ref{subsec:detobj}, the linking of these objects is discussed in Section~\ref{subsec:linking},
and the identification of groups of these links with particles is discussed in
Section~\ref{subsec:groupid}.

\begin{figure}[ht]
 \begin{center}
    \includegraphics[width=0.95\textwidth]{figures/objects/pf.pdf}
      \end{center}
\caption{A schematic of a slice of the CMS detector in the plane transverse to the beam line.
The trajectories of an electron, photon, charged hadron, neutral hadron, and muon are superimposed
with the interactions that each of these particles would have with the various subsystems.}
\label{fig:PF}
\end{figure}

\subsection{Detector Objects\label{subsec:detobj}}

The first step of the PF algorithm is to assemble detector objects created separately from
individual subsystems. In the tracker, hits in pixels or strips are associated into a track candidate
through the iterative Combinatorial Track Finder algorithm~\cite{PFPAS2009}. In the muon chambers,
standalone muon tracks (to distiguish from tracks formed by tracker hits)
are assembled from hits in the three muon detectors, accounting
for the nonuniform magnetic field and large detector budget and contraining a track candidate to
intersect the beamline. In the ECAL and HCAL, clustering of energy deposits is performed around a
cluster seed, which is identified as a detector unit with an amount of energy exceeding a
detector-dependent threshold and corresponding to a local maximum.
Units are added to adjacent seeds if their energy exceeds a noise threshold, and the PF clusters are
formed by redistributing the the energy back to the cluster seeds, recalculating the position as a
weighted average over the energy of each contributing cluster.

\subsection{Linking\label{subsec:linking}}

The next step of the PF algorithm is to link the detector objects to assemble PF candidates.
Possible links are between a track and standalone muon track, between a tracker track and a cluster, and
between two clusters, each having their own associated linking parameters. A link is formed between
a track and standalone muon track when the two can be merged into a global track with a fit
having $\chi^2$ below a threshold. A link is formed between a track and a cluster when the extrapolation of the track is within a certain distance of the cluster position. A link is formed between two clusters
(either both in ECAL, both in HCAL, or one in each) when the two clusters are within a certain distance.

\subsection{Grouping and Identification\label{subsec:groupid}}

An ensemble of links creates a block, and identification proceeds interatively through PF candidates.
First, muons are identified from those blocks that contain a global track having a momentum
sufficiently close to the momentum of the contained track. Then electrons are identified from blocks
containing a track and an ECAL cluster where the track and energy cluster satisfy requirements
consistent with the signature of an electron. Next photons and hadrons are identified from blocks
containing a track and a cluster from either ECAL or HCAL. If the calibrated energy in the clusters is
greater than the sum of the momentrum of the associated tracks, PF photons or PF neutrons are created
from the difference. If the difference is less than the energy in the ECAL clusters, a PF photon is
created from the block; if the difference is greater than the energy in the ECAL clusters,
a PF photon and a PF neutral hadron are made from the excesses ECAL and HCAL energy, respectively.
Finally, if the calibrated energy in the clusters is less than the sum of the momentrum of the associated
tracks, a search for fake tracks and additional muons in the block is performed, and what remains
in the block is a PF charged hadron.

From the list of PF candidates in an event, jets and taus are constructed by clustering nearby
hadrons. In this way, the clustering of PF hadrons represents the original quark or tau from the
underlying interaction. Missing transverse energy, which is a signature of one or more neutrinos
in the event and/or the mismeasurement of the energy of PF candidate, is obtained by
\begin{equation}
\met = - \sum_i \vec{p}_{{\rm T},i} \, ,
\end{equation}
where the sum is over all PF candidates in the event.

The treatment for photons is discussed in more detail in Section~\ref{sec:photons}. The construction
of jets from PF candidates is discussed in more detail in Section~\ref{sec:jets}. 


%Matching the muons to the tracks measured in the silicon tracker results in a transverse momentum
%resolution between 1 and 5\,\% for \pt values up to 1~TeV. The ECAL has an energy resolution better
%than 0.5\% for unconverted photons with transverse energies above 100~GeV.
%The HCAL, when combined with the ECAL, measures jets with a resolution
%$\Delta E/E \approx 100\,\% / \sqrt{E\,[\gev]} \oplus 5\,\%$.

\section{Photons\label{sec:photons}}

PF photon cadidiates are reconstructed from clustering individual units in the ECAL
and checking consistency with tracks.
The calorimeter signal are calibrated for several detector effects~\cite{CMS-PAS-EGM-10-005,ECALpaper},
providing the best energy resolution possible.
The energy scale is corrected in both data and simulation, while the photon energy is smeared in
simulation in order to reproduce the same energy resolution that is observed in data.

With the PF photon candidates in hand, additional requirements are imposed in order to further
separate prompt photons from fake photons originating from misidentified electrons or from jets.
These additional requirements include an electron veto, an upper threshold on the energy deposited
in HCAL in the region about the candidate, isolation, and the shower shape. The electron veto 
removes photons if there is an electron candidate matching the photon ECAL cluster with no missing
hits in the tracker and with no matching reconstructed conversion.
Isolation requirements place
thresholds on the amount of ECAL energy deposited in a region about the cluster; these are both
detector based and PF based. Requirements on the electromagnetic shower shape include the
width of the shower in terms of ECAL detector units and
ratio of the amount of energy in a 3x3 unit box around the cluster seed to that around a 5x5 unit box.

After the quality cuts, the two photons candidates are requested to satisfy the sliding asymmetric cuts
\begin{subequations}
\begin{equation}
p_{{\rm T},\gamma_1} > \frac{\Mgg}{3}
\end{equation}
\begin{equation}
p_{{\rm T},\gamma_2} > \frac{\Mgg}{4} \, ,
\end{equation}
\end{subequations}
where $p_{{\rm T},\gamma_1}$ and $p_{{\rm T},\gamma_2}$ are the transverse momenta of the
leading and subleading photons, respectively.
The fact that these requirements on the transverse momentum are scaled by the diphoton invariant mass
prevents turn-on effects which could distort the shape of the $\Mgg$ spectrum for low values of $\Mgg$.
Both photons have the requirement that their position be within the ECAL acceptance of
$\right|\eta\left| < 2.5$ with an exclusion on the ECAL gap between the barrel and endcaps
In the case that there are more than two photons passing the identification and kinematic requirements,
the two with the largest $p_{\rm T}$ are choosen. After this choice, the diphoton mass is required to
be between 100 and 180 GeV.

Figure~\ref{fig:mgg_onlyhiggs} shows the resulting simulated distributions for the diphoton mass 
of the resonant signal and resonant backgrounds.
In the case of the resonant background, the sum
of all SM $\Hgg$ production mechanisms is shown as a single contribution.
Figure~\ref{fig:mgg_controlplot} shows the same
distribution for data and the sum of background. (Note that these figures include requirements
on jets, discussed in Section~\ref{sec:jets}.)
The resolution of the diphoton mass spectrum for the signal is a few GeV.
The two primary drivers in this resolution are the energy resolution of the individual photons
and the direction of the photons from their origin, which is synonymous with the identification
of the vertex from which they were produced. Due to the presense of the $\Hbb$ decay in the search,
the tracks from the jets allow for the efficient identification of the correct vertex since photons
do not leave tracks unless they convert in the tracker.
The criterion for the vertex choice is that which has the maximimum $\sum_i p_{{\rm T},i}$, where
the sum is over all of the tracks associated to a particlar vertex. With this criterion,
the relative contribution to the diphoton mass resolution due to the vertex choice is
negligible with respect to the energy resolution for individual photons.

\begin{figure}[ht]
 \begin{center}
    \includegraphics[width=0.70\textwidth]{figures/objects/DiPhotonMass_OnlyHiggs.pdf}
      \end{center}
\caption{Simulated diphoton mass spectrum for the signal and the sum of all production mechanisms of the
SM Higgs boson after basic selections on photons and requesting at least one loose b--tagged jet.
The spectra is normalized to one.}
\label{fig:mgg_onlyhiggs}
\end{figure}

\begin{figure}[ht]
 \begin{center}
   \subfigure{\includegraphics[width=0.70\textwidth]{figures/objects/DiPhotonMass_ShapeNormalized_sys.pdf}}\\
   \subfigure{\includegraphics[width=0.70\textwidth]{figures/objects/DiPhotonMass_ShapeNormalized_Log_sys.pdf}}
      \end{center}
\caption{Control plots for the diphoton mass spectrum after basic photon and jet selections
and requiring at least one loose b--tagged jet. The simulation is normalized to data and
the statistical uncertainty on the number of simulated events is shown in dashed overlay.
The top (bottom) figure is in linear (log) scale.}
\label{fig:mgg_controlplot}
\end{figure}

\section{Jets\label{sec:jets}}



