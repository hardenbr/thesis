\chapter{Results\label{ch:results}}

After applying the preselection, event classification, and mass windows, signal efficiency
and event yields are examined. This is detailed in Section~\ref{sec:yields}. Then, signal yield
in data is evaluated through the calculation of the confidence level (CL)
for exclusion or for discovery of double Higgs production is evaluated from a simultaneous fit
to the $\Mgg$, $\Mggjjk$, or $\Mgg \times \Mjj$ specta for the low-mass resonant, high-mass resonant,
or nonresonant searches, respectively.
To compute the upper limits on the production cross section, the modified frequentist approach
$\text{CL}_s$ is used with an asymptotic approximation, taking thee profile likelihood as a test
statistic~\cite{CLS1,CLS2}. This calculation is discussed in Sections~\ref{sec:resresults} for the
resonant search and in Section~\ref{sec:nonresresults} for the SM nonresonant search.

\section{Signal Efficiencies and Yields\label{sec:yields}}

The signal effiency as a function of $m_X$ for the resonant search is summarized in
Figure~\ref{fig:eff_res}. The signal efficiency increases from 260 GeV to 900 GeV
from better photon and jet reconstruction efficiencies. The signal efficiency
peaks at and drops after 900 GeV due to the merging of the two jets from the decay $\Hbb$ into
a single jet. For future consideration in extending the search above 1.1 TeV,
through jet substructure techniques would be necessary to resolve the jet merging~\cite{Ellis:2009su}.
Both categories contribute approximately equally to the overall efficiency.

\begin{figure}[ht]
 \begin{center}
    \includegraphics[width=0.70\textwidth]{figures/results/eff_all.pdf}
      \end{center}
\caption{Signal efficiency for the resonant search for final selection.}
\label{fig:eff_res}
\end{figure}

The yields for the low-mass resonant search at $m_X = 300$~GeV are summarized in
Table~\ref{table:yield_lowmass_res}. Note that there is a normalization disagreement in the
$\gamma\gamma j$ and $\gamma j$ contributions as the simulation has limitations in modeling
QCD with one or two hard photons. As a result, the backgrounds are not added; the purpose is to
highlight relative contributions. The yields for the high-mass resonant search are summarized in
Table~\ref{table:yield_highmass_res}. For this search, the requirements are independent of
the mass hypothesis, so a several signal hypotheses can be shown together.

\begin{table}[ht]
  \centering
  \renewcommand{\arraystretch}{1.4}
  \caption{Event yields for the low-mass resonant search at 300 GeV. Expectations are given for
the signal, resonant background, and nonresonant background. Counts are given for data. Note that
there is a normalization disagreement coming from the shortcomings of simulating QCD with one
or two hard photons.}
  \begin{tabular}{|c|c|c|}
\hline
Sample & High purity & Medium purity\\
\hline
Radion (300$~$GeV, $\Lambda_R = 1$~TeV)  & 18.73 & 21.66     \\
\hline
ggF $\Hgg$                &  0.02  &  0.19 \\
VBF $\Hgg$                &  0.00  &  0.04 \\
$WH(\gamma\gamma)$        &  0.00  &  0.05 \\
$ZH(\gamma\gamma)$        &  0.00  &  0.03 \\
$t\bar{t}H(\gamma\gamma)$ &  0.10  &  0.15 \\
\hline
$\gamma\gamma j$                      & 8.9  &  188  \\
$\gamma j$                            & 0.00 &  9.2  \\ 
QCD                                   & 0.00 &  0.00 \\ 
$Z/\gamma^*(\ell^+\ell^-) + Z(\ell^+\ell^-)\gamma + W(\ell\nu)\gamma\gamma$ & 0.00 &  0.21 \\
$t\bar{t}\gamma\gamma + t\gamma\gamma + t\bar{t}\gamma j$ & 0.44 &  1.2  \\
\hline
Data                                  & 21 & 230 \\
\hline
\end{tabular}

  \label{table:yield_lowmass_res}
\end{table}

\begin{table}[ht]
  \centering
  \renewcommand{\arraystretch}{1.4}
  \caption{Event yields for the high-mass resonant search. Expectations are given for
the signal, resonant background, and nonresonant background. Counts are given for data. Note that
there is a normalization disagreement coming from the shortcomings of simulating QCD with one
or two hard photons.}
  \begin{tabular}{|c|c|c|}
\hline
Sample & High purity & Medium purity\\
\hline
Radion (500$~$GeV, $\Lambda_R = 1$~TeV)        &  6.08  & 6.47     \\
Radion (700$~$GeV, $\Lambda_R = 1$~TeV)        &  2.92  & 3.03     \\
Radion (1000$~$GeV, $\Lambda_R = 1$~TeV)       &  0.94  & 1.12     \\
\hline
ggF $\Hgg$                &  0.07  &  0.6  \\
VBF $\Hgg$                &  0.01  &  0.12 \\
$WH(\gamma\gamma)$        &  0.00  &  0.10 \\
$ZH(\gamma\gamma)$        &  0.03  &  0.07 \\
$t\bar{t}H(\gamma\gamma)$ &  0.24  &  0.50 \\
\hline
$\gamma\gamma j$                      & 3.0  &  70   \\
$\gamma j$                            & 0.00 &  3.0  \\
QCD                                   & 0.00 &  0.00 \\
$Z/\gamma^*(\ell^+\ell^-) + Z(\ell^+\ell^-)\gamma + W(\ell\nu)\gamma\gamma$ & 0.00 &  0.08 \\
$t\bar{t}\gamma\gamma + t\gamma\gamma + t\bar{t}\gamma j$ & 0.15 &  0.55 \\
\hline
Data                                  & 8 & 79 \\
\hline
\end{tabular}

  \label{table:yield_highmass_res}
\end{table}

The yields for the nonresonant search are summarized in Table~\ref{table:yield_nonres}.
The yields are higher in the nonresonant search because the $\Mggjjk$ spectrum is much less
disciminating than in the low-mass resonant search and because the $\Mjj$ is fit rather than
selected.

\begin{table}[ht]
  \centering
  \renewcommand{\arraystretch}{1.4}
  \caption{Event yields for the nonresonant search. Expectations are given for
the SM nonresonant signal, resonant background, and nonresonant background.
Counts are given for data. Note that
there is a normalization disagreement coming from the shortcomings of simulating QCD with one
or two hard photons.}
  \begin{tabular}{|c|c|c|c|c|}
\hline
 & \multicolumn{2}{c|}{High Purity} & \multicolumn{2}{c|}{Medium Purity} \\
Sample & high $\Mggjjk$ & low $\Mggjjk$ & high $\Mggjjk$ & low $\Mggjjk$ \\
\hline
$\gamma\gamma j$     &  90.9\% & 76.4\%  & 81.8\% & 82.6\% \\
$\gamma j$           & $<$~0.1\% & 15.6\% & 15.2\%  & 16.3\%  \\
QCD                  & $<$~0.1\% & $<$~0.1\% & $<$~0.1\% & $<$~0.1\% \\
$Z/\gamma^*(\ell^+\ell^-) + Z(\ell^+\ell^-)\gamma + W(\ell\nu)\gamma\gamma$
   & $<$~0.1\% & $<$~0.1\% & 1.2\% & 0.1\% \\
$t\bar{t}\gamma\gamma + t\gamma\gamma + t\bar{t}\gamma j$ &  9.1\% & 8.0\% & 1.8\% & 1.0\% \\
\hline
\end{tabular}

  \label{table:yield_nonres}
\end{table}

\section{Resonant Results\label{sec:resresults}}

The  95\% CL
expected and observed median upper limits are shown in Figure 8.  A zoom in the low mass
region only is shown on the Figure 9 (left) while the Figure 9 (right) shows the exclusion lim-
its for the high-purity category only.  This latest result is provided to simplify the comparison
with new physics models where the Higgs branching fraction to the photons and b quarks
can be modified with respect to the SM. The green and yellow bands represent the 1
s
and 2
s
confidence intervals around the expected limit.  In the Figure 8 the vertical dashed line indi


\section{Nonresonant Results\label{sec:nonresresults}}


%\section{The Future\label{sec:future}}

