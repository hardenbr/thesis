\chapter{Event Selection\label{ch:selection}}

This chapter discusses additional event-wide requirements imposed to further
improve the sensitiviy of selecting the signal of interest over the background contributions
beyond that of the preselection requirements covered in Chapter~\ref{ch:objects}.
Event classification is covered in Section~\ref{sec:classification}, in which two classes of
events are defined for extracting signal separately.
Section~\ref{sec:higgsreconstruction} discusses how the photons and jets passing the preselection
requirements are assembled into Higgs candidates. For the resonant search $X\rightarrow HH$,
Section~\ref{sec:Xreconstruction} discusses how the two Higgs come together to construct a resonant
candidate. Finally, the construction of a data control
sample, discussed in Section~\ref{sec:dataCS}, allows for the study of additional event-wide
requirements for improving sensitivity, discussed in Section~\ref{sec:optim}.

\section{Event Classification\label{sec:classification}}

B-tagging, presented in Section~\ref{subsec:btag}, is used at the preselection level through
requiring 1 CSVL b-tagged jet. To further improve the sensitivity, the working point used for
signal extraction is the medium one (CSVM), which corresponds to an efficiency of around 60--70\%
and a mistag rate 1--2\%, depending on the jet $p_{\rm T}$.
There is also a tight working point (CSVT) not used in this analysis,
which corresponds to a mistag rate of 0.1\%.
With the CSVM working point, events are classified into three categories
based on the number of CSVM b-tagged jets. Events with two or more b-tags, called high purity,
drive the sensitivity of the search. Events with one b-tag, called medium purity,
bring low contribution but allow for increased signal acceptance. Events without any b-tags,
called low purity, are only used for cross-checks and are rejected from the main analysis.
The preselection cuts and event categorization
are summarized in Table~\ref{table:gencut}.

\begin{table}[ht]
  \centering
  \renewcommand{\arraystretch}{1.4}
  \caption{Summary of the selection applied to photons and jets and the event classification.}
  \begin{tabular}{|c|c|c|}
\hline
Photons & Jets & Classification \\
\hline
tight photon identification                        & loose jet identification  & \\
$p_{T \gamma_1}/ \Mgg > 1/3$  & pileup rejection   & \\
$p_{T \gamma_2}/ \Mgg > 1/4$  & $p_{T j} >25$~GeV  & \\
$|\eta_{\gamma}|<2.5$                              & $|\eta_{j}|<2.5$  & $\ge$~2 CSVM b-tags \\
$ 100 < \Mgg < 180$~GeV                 & $\ge$~1 CSVL b-tag & exactly 1 CSVM b-tag \\
\hline
\end{tabular}

  \label{table:gencut}
\end{table}

\section{Higgs Reconstruction\label{sec:higgsreconstruction}}

From the lists of photons and jets passing the identification and kinematic requirements, Higgs
candidates 

talk about photon choice, jet choice

The  Higgs  boson  candidate  is  reconstructed  in  the  medium-purity  events  by  pairing  the  b-
tagged jet with the non b-tagged jet giving largest
p
T
of the pair.  For the high-purity events
the pair of b-tagged jets giving the largest
p
T
for the Higgs candidate is used.  This procedure
selects the correct jets in more than 80% of the cases and does not produce any local peaking
structure in the background.
The resulting signal mass shape of the Higgs candidate decaying into two b-quarks is shown on
the Figure 1 (top-right). For the high-purity category, the typical resolution, defined as the half-
width at half-maximum, decreases from 20 GeV at
m
X
=
300 GeV to 15 GeV at
m
X
=
1 TeV. In
the medium-purity category, the resolution is worse, decreasing from 25 GeV at
m
X
=
300 GeV
to 15 GeV at
m
X
=
1 TeV.



\section{Resonance Reconstruction and Kinematic Fit\label{sec:Xreconstruction}}

\section{Data Control Sample\label{sec:dataCS}}

\section{Optimization Studies\label{sec:optim}}
Separate resonant and nonresonant here


