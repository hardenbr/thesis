\chapter{Event Selection\label{ch:selection}}

This chapter discusses additional event-wide requirements imposed to further
improve the sensitiviy of selecting the signal of interest over the background contributions
beyond that of the preselection requirements covered in Chapter~\ref{ch:objects}.
Event classification is covered in Section~\ref{sec:classification}, in which two classes of
events are defined for extracting signal separately.
Section~\ref{sec:higgsreconstruction} discusses how the photons and jets passing the preselection
requirements are assembled into Higgs candidates. For the resonant search $X\rightarrow HH$,
Section~\ref{sec:Xreconstruction} discusses how the two Higgs come together to construct a resonant
candidate. Finally, the construction of a data control
sample, discussed in Section~\ref{sec:dataCS}, allows for the study of additional event-wide
requirements for improving sensitivity, discussed in Section~\ref{sec:optim}.

\section{Event Classification\label{sec:classification}}

B-tagging, presented in Section~\ref{subsec:btag}, is used at the preselection level through
requiring 1 CSVL b-tagged jet. To further improve the sensitivity, the working point used for
signal extraction is the medium one (CSVM), which corresponds to an efficiency of around 60--70\%
and a mistag rate 1--2\%, depending on the jet $p_{\rm T}$.
There is also a tight working point (CSVT) not used in this analysis,
which corresponds to a mistag rate of 0.1\%.
With the CSVM working point, events are classified into three categories
based on the number of CSVM b-tagged jets. Events with two or more b-tags, called high purity,
drive the sensitivity of the search. Events with one b-tag, called medium purity,
bring low contribution but allow for increased signal acceptance. Events without any b-tags,
called low purity, are only used for cross-checks and are rejected from the main analysis.
The preselection cuts and event categorization
are summarized in Table~\ref{table:gencut}.

\begin{table}[ht]
  \centering
  \renewcommand{\arraystretch}{1.4}
  \caption{Summary of the selection applied to photons and jets and the event classification.}
  \begin{tabular}{|c|c|c|}
\hline
Photons & Jets & Events classification \\
\hline
tight photon identification                        & loose jet identification  & \\
$p_{T \gamma_1}/ \Mgg > 1/3$  & pileup rejection   & \\
$p_{T \gamma_2}/ \Mgg > 1/4$  & $p_{T j} >25$~GeV  & \\
$|\eta_{\gamma}|<2.5$                              & $|\eta_{j}|<2.5$  & $\ge$2 CSVM b-tagged jets \\
$ 100 < \Mgg < 180$~GeV                 & at least 1 CSVL b-tagged jet & exactly 1 CSVM b-tagged jet \\
\hline
\end{tabular}

  \label{table:gencut}
\end{table}

\section{Higgs Reconstruction\label{sec:higgsreconstruction}}

From the lists of photons and jets passing the identification and kinematic requirements, Higgs
candidates are constructed. When there are more than two photon or jet candidates, a choice
must be made as to which pair is consistent with the decay of a Higgs. In the case
of chosing the diphoton candidate, the two photons with the highest $p_{\rm T}$ are chosen
as jets faking photons tend to dominate for softer photon candidates.

In the case of the dijet candidate, four choices were considered after b-tagging:
\begin{itemize}
\item the two jets with the highest $p_{\rm T}$,
\item the pair of jets that maximizes $p_{{\rm T},jj}$,
\item the pair of jets that maximizes $\frac{p_{{\rm T},jj}}{\Mjj}$, and
\item the pair of jets which minimizes $|\Mjj-\Mgg|$.
\end{itemize}
In the 2-tag category, the choice is among all b-tagged jets, while in the 1-tag category, the
choice is among all non-tagged jets to pair to the b-tagged one.
The criterion used in the analysis is the one that maximizes $p_{{\rm T},jj}$. 
This choice selects the correct jets roughly 87\% of the time, and it does not produce any local peaking
structure in the background. For the resonant search
in the high-purity (medium-purity) category, the resolution, defined as the half-width at half-maximum,
decreases from 20 (25) GeV to 15 (15) GeV as the resonance mass increases from 300 GeV to 1 TeV.
This was also shown visually in Figure~\ref{fig:mjj_onlyhiggs}.

\section{Resonance Reconstruction and Kinematic Fit\label{sec:Xreconstruction}}



\section{Data Control Sample\label{sec:dataCS}}

\section{Optimization Studies\label{sec:optim}}
Separate resonant and nonresonant here


