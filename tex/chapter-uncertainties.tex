\chapter{Systematic Uncertainties\label{ch:uncertainties}}

The expected number of signal events is estimated using simulation.  Possible discrepancies
in the photon or jet reconstruction and identification, as well as differences in the b-tagging
efficiency between data and Monte Carlo are corrected through data-to-simulation scale factors.
The experimental uncertainties are applied to the reconstructed objects in simulated events by
scaling and smearing the relevant observables.  The total normalisation uncertainty related to
the uncertainty in the estimation of the produced luminosity is taken to be 2.6\% [36].  The
other sources can be separated in three categories: the photon-related and jet-related and theory related.

\section{Photon Uncertainties\label{sec:photonunc}}

The photon-related uncertainties are taken from [22].  An uncertainty between 0.23 and 0.93%
is considered on the energy resolution (PER) and between 0.12 and 0.88% on the energy scale
(PES),  depending  on
h
g
and  the  electromagnetic  shower  shape.   When
p
T
g
>
100  GeV  the
uncertainty of the energy scale is conservatively increased up to 1%.  A 1% normalization un-
certainty is assumed on the offline photon selection efficiency and on the trigger efficiency. An
additional conservative normalization uncertainty of 5% is assumed for the high mass region
to account for the differences in the p
T
spectrum of the signal photons and of the electrons from
Z
!
ee used to estimate the quoted uncertainties.

\section{Jet Uncertainties\label{sec:jetunc}}

The jet energy scale uncertainty (JES) is accounted for by varying the jet response by 1-2%,
depending on the kinematics [31], while the jet energy resolution uncertainty (JER) by varying
the jet resolution by 10%. An additional 1% uncertainty on the 4-body mass response accounts
for effects in the high-mass region related to the partial overlap between the two b-jets coming
from the Higgs decay. The uncertainty on the b-tagging efficiency is estimated by varying the b-
tagging scale factor by one standard deviation in each category [33], and the related systematics
were shown to be anti-correlated between the two categories.

%\section{Other Experimental Sources\label{sec:otherunc}}

\section{Theory Uncertainties\label{sec:theoryunc}}

Theory systematics are considered for the SM single Higgs contribution including scale depen-
dence to account for the missing order effects and the dependency on proton parton density
functions [37, 38]. On the other hand, no theory systematics are assumed on the signal. Finally
an additional systematic uncertainty is assigned to the Higgs mass, both for the signal and for
the SM Higgs background.   It is taken to be 0.45 GeV and corresponds to the experimental
uncertainty from the Higgs mass measurement in the
H
!
ZZ
channel [15].

\section{Summary and Impact on Analysis\label{sec:uncimpact}}

The impact of the quoted systematic uncertainties on the result is summarized in
Table~\ref{table:systematics}.
The analysis is statistics limited, and the systematic uncertainties worsen the expected limits
by 1.7% at most. (3.8 for nonres)

\begin{table}[ht]
  \centering
  \renewcommand{\arraystretch}{1.4}
  \caption{Systematic uncertainties organized by search strategy.}
  \begin{tabular}{|c|c|}
\hline
\multicolumn{2}{|c|}{Common normalization uncertainties} \\
\hline
Luminosity & 2.6\%\\
Diphoton trigger acceptance & 1.0\% \\
\hline
\hline
\multicolumn{2}{|c|}{fit $M_{\gamma\gamma}$ - Background dominated} \\
\hline
\hline
\multicolumn{2}{|c|}{Normalization uncertainties} \\
\hline
Photons selection acceptance & 1.0\% \\ 
"b-tag" eff. uncertainty 2 btag cat & 4.6\% \\  
"b-tag" eff. uncertainty 1 btag cat & -1.2\% \\  
$M_{jj}$ and $p_{T, j}$ cut acceptance ( JES \& JER) & 1.5\%\\
$M_{\gamma\gamma\rm jj}$ cut acceptance (PES $\oplus$ JES \& PER  $\oplus$ JER) & 2\%\\
\hline
\multicolumn{2}{|c|}{Shape uncertainties} \\
\hline
Parametric scale shift (PES$\oplus$M(H) uncertainty)      & $\frac{\Delta M_{\gamma \gamma}}{M_{\gamma \gamma}} = 0.45 \oplus 0.35$\%\\
Parametric resolution shift (RES) & $\frac{\Delta \sigma}{M_{\gamma \gamma}} = 0.25$\% \\
                                  & $\frac{\Delta \sigma}{\sigma_{\gamma \gamma}} = 22$\% \\
\hline
\hline
\multicolumn{2}{|c|}{fit $M_{\gamma\gamma\rm jj}$ - Background dominated} \\
\hline
\hline
\multicolumn{2}{|c|}{Normalization uncertainties} \\
\hline
Photons selection acceptance & 1.0\% \\ 
"b-tag" eff. uncertainty 2 btag cat & 5.3\% \\  
"b-tag" eff. uncertainty 1 btag cat & -1.8\% \\  
$M_{jj}$ and $p_{T, j}$ cut acceptance ( JES \& JER) & 1.5\%\\
$M_{\gamma\gamma}$ cut acceptance (PES \& PER ) & 0.5\% \\
Extra High pt norm. uncertainty & 5.0\% \\
\hline
\multicolumn{2}{|c|}{Shape uncertainties} \\
\hline
Parametric abs. shift (PES $\oplus$ JES ) & $\frac{\Delta M_{\gamma \gamma {\rm jj} }}{M_{\gamma \gamma {\rm jj} }} = 0.45 \oplus (0.8 \oplus 1.0) = 1.4$\% \\
Parametric shift (PER $\oplus$ JER ) & $\frac{\Delta \sigma}{\sigma_{\gamma \gamma {\rm jj} }} = 10$\% \\
\hline
\end{tabular}

  \label{table:systematics}
\end{table}
