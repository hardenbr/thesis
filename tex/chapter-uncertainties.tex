\chapter{Systematic Uncertainties\label{ch:uncertainties}}

The expected signal yield is estimated through simulation which is corrected
due to limitations in the simulation to exactly produce what is observed in data.
Corrections imposed here are related to the photon and jet reconstruction and identification
and related to b-tagging.
The uncertainties associated with these corrections are applied to the reconstructed objects in
the simulation through scaling and smearing the observables of interest.
For the recorded luminosity, the normalization uncertainty is 2.6\%~\cite{CMS-PAS-LUM-13-001}.
The other sources are separated in terms of uncertainties related to photons, jets, and theory,
which are covered in Sections~\ref{sec:photonunc}, \ref{sec:jetunc}, and ~\ref{theoryunc},
respectively.

\section{Photon Uncertainties\label{sec:photonunc}}

The photon-related uncertainties consist of those pertaining to the photon energy resolution (PER)
and the photon energy scale (PES)~\cite{CMS-PAS-HIG-13-001}. As a function of the
electromagnetic shower shape and $\eta_\gamma$,
an uncertainty between 0.23 and 0.93\% is imposed on the PER,
and an uncertainty between 0.12 and 0.88\% on the PES. For hard photons, namely those with
$p_{{\rm T},\gamma} > 100$~GeV, the uncertainty on PES is increased to 1\%.
%These uncertainties are propogated through to the spectra of interest. For the $\Mgg$ spectrum,
%they give  uncertainties on the position and spread of the signal shape. For the
%$\Mggjjk$ spectrum, they give uncertainties on both the position and spread of the signal shape
%and on the normalization.

The photon preselection efficiency contributes a 1\% normalization uncertainty to the $\Mgg$
spectrum. The diphoton trigger efficiency contributes a 1\% normalization
uncertainty to all spectra.
An additional normalization uncertainty of 5\% is imposed in the high-mass resonant search
to account for the differences in the $p_{\rm T}$ spectrum between photons of the signal
and the electrons from $Z\rightarrow e^+ e^-$ used to estimate the PES and PER and their
corresponding uncertainties.

\section{Jet Uncertainties\label{sec:jetunc}}

The jet energy scale (JES) uncertainty is found by varying the jet $p_{\rm T}$ by 1--2\%,
depending on the jet $p_{\rm T}$ and $\eta$~\cite{JINST6}.
The jet energy resolution (JER) uncertainty is found by varying the jet resolution by 10\%.
In the high-mass search, the jets tend to have a higher boost and be closer together,
and effects related to their partial overlap are accounted for with an additional uncertainty of 1\%.
For the b-tagging efficiency uncertainty, the b-tagging scale factors are varied by one standard
deviation in each category~\cite{BTV}. 
The uncertainty for the b-tagging efficiency between the two categories
was found to have negative correlation.

\section{Theory Uncertainties\label{sec:theoryunc}}

No theory uncertainties are imposed for the resonant or nonresonant signal.
For the resonant background, theory uncertainties are imposed on the SM Higgs contribution. These
include contributions from missing order effects and the dependency on proton parton density
functions~\cite{Dittmaier:2011ti,Heinemeyer:2013tqa}.
A systematic uncertainty is imposed on the Higgs mass for both for the resonant or nonresonant signal
and for the resonant background.
This uncertainty of 0.45 GeV is taken from the Higgs mass measurement performed at CMS in the
$H\rightarrow ZZ \rightarrow 4\ell$ channel~\cite{Chatrchyan:2013mxa}.

\section{Summary and Impact on Analysis\label{sec:uncimpact}}

The impact of the quoted systematic uncertainties on the result is summarized in
Table~\ref{table:systematics}.
The analysis is statistics limited, and the systematic uncertainties worsen the expected limits
by at most 1.7\% (3.8\%) in the resonant (nonresonant) search.

\begin{table}[ht]
  \centering
  \renewcommand{\arraystretch}{1.4}
  \caption{Systematic uncertainties organized by search strategy.}
  \begin{tabular}{|c|c|}
\hline
\multicolumn{2}{|c|}{Common normalization uncertainties} \\
\hline
Luminosity & 2.6\%\\
Diphoton trigger acceptance & 1.0\% \\
\hline
\hline
\multicolumn{2}{|c|}{fit $M_{\gamma\gamma}$ - Background dominated} \\
\hline
\hline
\multicolumn{2}{|c|}{Normalization uncertainties} \\
\hline
Photons selection acceptance & 1.0\% \\ 
"b-tag" eff. uncertainty 2 btag cat & 4.6\% \\  
"b-tag" eff. uncertainty 1 btag cat & -1.2\% \\  
$M_{jj}$ and $p_{T, j}$ cut acceptance ( JES \& JER) & 1.5\%\\
$M_{\gamma\gamma\rm jj}$ cut acceptance (PES $\oplus$ JES \& PER  $\oplus$ JER) & 2\%\\
\hline
\multicolumn{2}{|c|}{Shape uncertainties} \\
\hline
Parametric scale shift (PES$\oplus$M(H) uncertainty)      & $\frac{\Delta M_{\gamma \gamma}}{M_{\gamma \gamma}} = 0.45 \oplus 0.35$\%\\
Parametric resolution shift (RES) & $\frac{\Delta \sigma}{M_{\gamma \gamma}} = 0.25$\% \\
                                  & $\frac{\Delta \sigma}{\sigma_{\gamma \gamma}} = 22$\% \\
\hline
\hline
\multicolumn{2}{|c|}{fit $M_{\gamma\gamma\rm jj}$ - Background dominated} \\
\hline
\hline
\multicolumn{2}{|c|}{Normalization uncertainties} \\
\hline
Photons selection acceptance & 1.0\% \\ 
"b-tag" eff. uncertainty 2 btag cat & 5.3\% \\  
"b-tag" eff. uncertainty 1 btag cat & -1.8\% \\  
$M_{jj}$ and $p_{T, j}$ cut acceptance ( JES \& JER) & 1.5\%\\
$M_{\gamma\gamma}$ cut acceptance (PES \& PER ) & 0.5\% \\
Extra High pt norm. uncertainty & 5.0\% \\
\hline
\multicolumn{2}{|c|}{Shape uncertainties} \\
\hline
Parametric abs. shift (PES $\oplus$ JES ) & $\frac{\Delta M_{\gamma \gamma {\rm jj} }}{M_{\gamma \gamma {\rm jj} }} = 0.45 \oplus (0.8 \oplus 1.0) = 1.4$\% \\
Parametric shift (PER $\oplus$ JER ) & $\frac{\Delta \sigma}{\sigma_{\gamma \gamma {\rm jj} }} = 10$\% \\
\hline
\end{tabular}

  \label{table:systematics}
\end{table}
