
\chapter{Introduction\label{ch:intro}}

Our understanding of fundamental particles and interactions has progressed much from the early days.
These beginnings could be with Empdocles and his four roots fire, earth, air, and water, or with
Aristotle relating these four roots to two of the four sensible quantities hot, dry, wet, and
cold~\cite{0415078547} as in Figure~\ref{fig:aristotle}
(not to omit classical elements from other philosophies and worldviews), or more recently with
John Dalton's atoms~\cite{dalton}. Or perhaps particle physics began with
the discovery of the electron by J.J. Thomson in 1897~\cite{thomson:electron},
which to this day has not been observed
to have internal structure or decay, with upper (lower) bounds on the radius (lifetime) of
10$^{-22}$~m (10$^{26}$~years)~\cite{1988PhST...22..102D,2002PhLB..525...29B}.

\begin{figure}[ht]
 \begin{center}
    \includegraphics[width=0.90\textwidth]{figures/intro/Four_elements_representation.png}
      \end{center}
\caption{The four roots how they relate to the sensible quantities.}
\label{fig:aristotle}
\end{figure}

Today we have the Standard Model (SM), the theoretical framework that best describes the
experimentally oberserved phenomena of the fundamental particles and their interactions. The theory
is not perfect, and it remains an overarching theme of particle physics to unify physical processes
at all energy scales under one single framework, if such a thing can be done at all.
The SM, its successes, and its shortcomings are described in
Sections~\ref{sec:SM}, \ref{sec:SMsuccess}, and \ref{sec:SMshortcomings}.

Recently in 2012, the last piece to the SM was put into place with the discovery of the Higgs boson.
This discovery, described in Section~\ref{sec:discovery}, is the foundation for the work based on
this thesis, the goal of which is to describe the first search for diHiggs production, a process
in which two Higgs bosons are produced. The motivations for what the search for this process means
in the context of SM physics and ``new'' physics is given in Section~\ref{sec:diHiggs}.

Finally, for those readers who have by chance come across this thesis and do not
have any physics training, Appendix~\ref{ch:mom} may be especially appealing.

\section{The Standard Model\label{sec:SM}}

\section{Higgs Discovery\label{sec:discovery}}
reference by sec:CMS

\section{Successes of the SM\label{sec:SMsuccess}}

\section{Shortcomings of the SM\label{sec:SMshortcomings}}
reference by sec:CMS


\section{diHiggs as a probe of SM and New Physics\label{sec:diHiggs}}

